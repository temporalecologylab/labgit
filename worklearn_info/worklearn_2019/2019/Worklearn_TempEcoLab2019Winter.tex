\documentclass[11pt, oneside]{article}   	% use "amsart" instead of "article" for AMSLaTeX format
\usepackage{geometry}                		% See geometry.pdf to learn the layout options. There are lots.
\geometry{letterpaper}                   		% ... or a4paper or a5paper or ... 
\usepackage{graphicx}				% Use pdf, png, jpg, or eps§ with pdflatex; use eps in DVI mode
								% TeX will automatically convert eps --> pdf in pdflatex		
\usepackage{amssymb}


% Some info to remember (courtesy of Mary O'Connor):
% Be sure to fill out each section thoughtfully and in way that indicates the job will train the student, be a fulfilling experience for them, but also NOT be too technical
% Do not use jargon
% Be sure supervision includes time with the PI
% Be sure to connect how the student's work will contribute to the lab's research goals
\begin{document}
\noindent \textbf{\Large{Work Learn Application: Winter 2019}}
% 2019 app draft by Deirdre Loughnan and Lizzie

\section {Job Description} 
Project Worker for the Temporal Ecology Lab (non-union: CUPE 116)

\subsection {Position summary, including lab's goals}
The Temporal Ecology Lab focuses on how global climate change is altering plant communities locally, regionally and globally. Recent research in our lab has examined how climate change shifts have advanced the timing of plants' leafout and flowering across diverse ecosystems. This academic year work will focus on the leafout phenology and related ecology of BC forests. The laboratory is hiring undergraduate students to join the lab and directly contribute to the projects of graduate students, as well as collaborative projects in the lab. \\

The student's duties and responsibilities will include:

\begin{enumerate}
\item Assisting in lab experiment on woody plant phenology by observing and recording phenological development, helping fill and transfer experimental units across growth chambers
\item Accompanying the lab on trips to collect woody plant cuttings in the field (mainly Manning and Babine Mountain Provincial parks in October 2019)
% \item Identifying woody tree and shrub species (after training and with close supervision)
\item Drying, weighing and taking the volume of leaves and woody tissue to quantify functional traits of diverse woody species from BC forests
\item Learning and implementing basic data quality standards for entering and cleaning data collected
\end{enumerate}

\section {Job Description}
Project Worker: non-union (CUPE 116)

% Job Description online in simplicity should include: 'Position summary, including lab's goals & duties', responsibility and complexity, Supervision ... be sure to format neatly online!

\subsection {Complexity}

The student is expected to learn to identify the phenology of roughly 20 woody tree and shrub species, with extensive help from the lab and using keys developed in the lab (these project-specific keys make it fairly straight-forward to narrow down species and phenological identification quickly and provide photos of the species taken at our actual field sites and in similar experiments in our lab, making matching photos to plants in the field/lab much easier). Generally a graduate student will always be available to check and confirm any identifications. The student will also quantify leaf traits, and is expected to use basic measurement tools (i.e., measuring tapes, electronic scales) with consistency and precision. They will also need to learn how to use a simple flatbed computer scanner and basic imaging software to quantify leaf areas. 

These tasks require attention, training and focus but are generally not overly complex, once trained. All students trained in these methods in our lab in the past have been able to master them after training, and the skills are transferrable to other lab and field activities. For all these tasks work-learn students will be provided with step-by-step instructions and will have constant training and supervision for the first few days of any tasks. Someone is always available to answer questions and provide additional guidance. 

This student researcher's role is critical to the goals of the lab, and they will help determine how diverse woody tree and shrub co-occur in BC forests and how these plant communities will change in the future. The lab has a history of documenting and understanding climate-change induced changes in plant leafout and performance, which continues with this project. The data produced through this work-learn experience will directly contribute to lab research, including presentation, data and manuscript publications. 

\subsection {Supervision}
The student will be directly supervised by a graduate student, both in the field on data collection trips and while qualifying plant traits and helping meet data quality standards in the lab. The graduate student will provide an active and engaging environment for the work-learn student and will meet with the student regularly (often every day depending on the activity, but at least twice/week otherwise) and will be easily available to answer questions as they come up. The principle investigator (Dr. Wolkovich) will also meet with the work-learn student throughout the term (at the start and end of each term and otherwise every two weeks or as needed, whichever is more frequent). The goals of the meetings with Dr. Wolkovich are multiple and include: (1) to discuss the work-learn student's and lab's current goals, (2) check-in regarding training, review methods as needed, and provide feedback to the student as work proceeds, (3) handle any issues that come up and require discussion/supervision and, (4) at the end of the work-learn period discuss continuing opportunities in the lab and the student researcher's longer term (e.g., career) goals. Additionally Dr. Wolkovich will be available for informal discussion and mentoring during field and lab work periods, and these periods provide an opportunity for Dr. Wolkovich, graduate students and work-learn student researchers to work alongside one another in the lab and field. 

The work-learn student will be invited to participate in lab meeting and other lab activities, such as journal discussions. Through this position, the student will have the opportunity to better understand the process of conducting ecological research, including experimental design, data collection and archiving, and analysis, while also building diverse professional connections and skills through interactions with researchers and national park employees at field sites.  

% Qualifications online in simplicity should include: Qualifications

\subsection {Qualifications}
The preferred candidate will be an undergraduate and have a basic understanding of ecology and/or physiology in plants---previous experience in a lab or field setting is helpful, though this is not required provided the student has an enthusiasm to learn. A basic background in biology with coursework at the third year or higher is preferred, but also not required. 

The biggest qualification for a student is enthusiasm to learn and engage with the research topic. In addition, students who would be a good fit for the position and lab will have strong organization and communication skills, be detail oriented, work well alone and in a team, be interested in building lab and field skills. Students will gain skills in plant and phenological identification and plant trait methods, which are important introductory skills for future work in most plant-focused ecology and physiology labs. Additionally these basic skills can be transferred to other lab and field environments. 

\section {Student Learning Components}

\subsection {Orientation}
In addition to all New Employee training required by UBC, the student will benefit from a suite of additional orientations by the graduate student and PI (Wolkovich). These include an introduction to lab protocols (review of all taxonomic materials, the project specific plant-guide (taxonomy and phenology guides), basic measuring devices, field tools, computer software and scanner), an overview of the lab's safety protocols while in the lab and field, as well as general lab and field `good practices,' and an overview of the research project and related biological concepts. These orientations are spread out over the first days to maximize student uptake, but students can always ask questions later as they come up (as the graduate student and PI are very often available) or in regularly scheduled meetings. Additionally, the lab maintains a suite of protocols that review the step-by-step tasks required for the various methods, and the lab runs `scrum' once a week when general questions are posed and answered by all in the lab.

\subsection{Mentorship, Feedback, and Support}
% Follow Mary's section: (1) Feedback & Support, (2) Mentorship opportunities and (3) Encouragement (though maybe we can call this Encouragement \& Future Opportunities)
\noindent \emph{Feedback \& Support:} The graduate student will work alongside the student (and at times PI) in the field and lab, who is always happy to answer questions and provide support. Through regular meetings (at least every two weeks) with the PI, the student will be provided with frequent constructive feedback and opportunities to communicate any challenges or areas in which additional support is needed to ensure the student's success. In addition to regular meetings and interaction while working together in the field, the PI and graduate student are generally always available to answer questions as they come up. The group maintains an open and accessible atmosphere that strives to encourage questions, and informal opportunities for conversations and feedback.

\noindent \emph{Mentorship:}
Direct mentorship comes from the graduate student and PI Wolkovich, who will oversee the student researchers in the lab, field and other research settings. Through regular (scheduled) and informal meetings the two mentors will provide feedback on progress and skills to the student, discuss current research goals and also review the student's longer-term goals (we do this especially at the end of the work-learn period when the student has had exposure to the research environment). Indirect mentorship is possible through other graduate students and postdoctoral associates in the lab and we always encourage student researchers in our lab interested in pursuing ecology as a career to query all lab members about their career paths and general advice. Finally, the PI will also conduct a mid-point performance evaluation, as recommended by Career Services, and will meet with the student to review the evaluation near the end of the work-learn term.

The student researcher is also welcomed at lab meetings (generally weekly during term) and at weekly scrum. Scum is a weekly meeting to review what everyone in the lab is working on and to bring general questions and challenges to the group. In this way work-learn student researchers can ask questions, but also see the hurdles and issues everyone in the lab faces. Lab meeting and scrum also provide a mechanism for the student to learn about the diversity of research programs in the lab. 

% \emph{Encouragement \& Future Opportunities:}

\subsection{Personal, Professional, \& Academic Development} 
Research thrives on new energy from the next generation of researchers, thus our lab welcomes student researchers into our group and seeks to give them a full experience of the process of research. We want students to experience research from idea generation to lab and field work, data entry, analysis and also have a sense of the next stages (manuscript preparation, translating research into practice). Thus, the student researcher will be encouraged to participate in lab activities, including lab meetings, weekly scrum (detailed above) and journal discussions, as well as to contribute intellectually to ongoing research. By including the student in discussions of research methods, challenges, and methods to improve efficiencies, they will see first hand the considerations and reasoning behind experiment design and research. 

This position is designed to help the student develop professionally and personally. Students learn how to communicate and work with land and lab managers and the community as we collect data in public areas, national parks and in the growth chamber facilities in Forestry. Through the field work, the student will become familiar with the woody species of British Columbia's forest ecosystem and gain a deeper understanding of how forests are changing with climate change. In the lab, the student will learn to quantify key plant traits, developing their understanding of the relationship between plant morphology and ecological theory, while also gaining experience in data collection and proper data management. In participating in these activities, the students will be able to build on their course material by gaining knowledge of the diversity and ecology of local forests that compliments the course curriculum in many upper-year forestry courses. Their inclusion in lab meetings and journal discussions will allow them to develop the critical thinking and active reading skills required by upper-year classes and graduate studies. Additionally, through discussions with all lab members the student can learn about career paths in ecology, how diverse those paths can be, and in turn we in the lab aim to learn more about the student and their interests and goals. 

We especially recognize that those goals and interests can change over time and thus keep discussions open and continual throughout the work-learn term. While many student researchers in our lab have gone on to MSc or PhD programs in ecology or related fields, or have taken up conservation or related land management jobs, some have gone on to start their own record labels, work for conservation law firms or work on media arts, and we have worked to embrace all these interests as they fit within the context of the lab (e.g., the media arts lab member refined our photos of phenological events and helped build some of our nicest in-house keys). 

Students whose interests are well matched to the lab's research goals at the end of their work-term are encouraged to continue in the lab. This can mean continuing to collaborate on the same project, developing a project of their own or switching onto other lab projects they are excited about. We also are happy to connect students with other labs as their interests require. For those most excited about the lab's research we support honours projects and past honours students are now field technicians and PhD students in ecology, with one out of three publishing their results in a peer-reviewed journal. Overall the Temporal Ecology lab has supported almost two dozen junior researchers, with the vast majority going on to careers or degree programs in ecology and conservation. These former lab members show not only the success of our program, but also form a network future work-learn students can use to explore career opportunities. 

\end{document}  