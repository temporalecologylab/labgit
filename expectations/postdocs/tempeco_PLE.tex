\documentclass[11pt,a4paper,oneside]{article}
\renewcommand{\baselinestretch}{1}
% \renewcommand*{\thefootnote}{\fnsymbol{footnote}}
\usepackage{sectsty,setspace} 
\usepackage[top=1.00in, bottom=1.0in, left=1in, right=1in]{geometry} 
\usepackage{graphicx}
\usepackage{epstopdf}
\usepackage{amsmath,latexsym,amssymb,wasysym}
\usepackage{natbib}
\usepackage{hyperref}

\usepackage{fancyhdr}
\pagestyle{fancy}
\fancyhead[LO]{Lab expectations}
\fancyhead[RO]{Wolkovich Lab}

\newenvironment{smitemize}{
\begin{itemize}
  \setlength{\itemsep}{1pt}
  \setlength{\parskip}{0pt}
  \setlength{\parsep}{0pt}}
{\end{itemize}
}

\def\labelitemi{--}

\begin{document}
\bibliographystyle{/Users/Lizzie/Documents/EndnoteRelated/Bibtex/styles/amnat}


\title{Great Expections\\ for the Wolkovich Lab}
\date{\today}
%\author{}
\maketitle
% \tableofcontents

\section{Why are you giving me this?}
The Temporal Ecology Lab (aka `the Wolkovich lab') is fortunate enough to be a pretty happy, productive and efficient place to be most of the time (IMHO). The reason for this is the people---who are generally amazing team members. \\

\noindent Sometimes, how to be an amazing team member is not obvious to everyone though. So to be clear, I have written up a list of what I expect. Most of these tasks should be easy for you and effectively are my way of explaining what I see as a good lab member. That said, everyone is different and many things can come up over someone's time in the lab, so think of these as general guidelines and if you find times in your career you cannot meet all the expectations---just reach out to me (Lizzie) to discuss. 


\section{Lab expectations}
\begin{itemize}
\item Attend lab meetings (happily). % Bring snacks when asked or for any special occasion you think of.
\item Attend scrum and give weekly updates. 
\item Follow the lab's data management plan, including how you manage metadata and in backing up all your data.
\item Schedule meetings with Lizzie as needed. Please ask for help when you need it.
\item Plan for a minimum one-week turnaround on grants, drafts, letters etc. given to or requested from Lizzie. Depending on the request and Lizzie's schedule it could be two weeks (or when slammed with revisions, plus drafts etc. it has stretched beyond two weeks). 
\item Exploit the seminar resources around you. Aim to attend at least one talk a week of the following: Biodiversity Seminar Series (Wednesdys at noon); Faculty of Forestry Seminar series (not every week, but Tuesdays at noon); Departmental Seminar series (Forest \& Conservation Sciences); Botany Seminar Series (Tuesdays at 12:30)  % Arboretum seminar series (Mondays at 12:10), Herbaria (Tuesdays at 12:07), OEB (some Thursdays at 4pm), HUCE. 
\item Attend all talks by lab-invited speakers and sign up to meet with them. 
\item Help out as requested in the lab. The lab has a variety of resources/events that make the lab happier and more productive but keeping these things running requires small commitments from all lab members. 

During your time in the lab you'll be asked to pitch in on various lab tasks and manage various resources. Some examples include: keeping the lab vehicle on top of its inspections; helping design, collect seeds or aid big experiments or long-term research; bring snacks to lab meeting, helping schedule invited speakers, etc.. Please help when asked to contribute to keeping the lab running well. 
\item Help out in other ways, not always requested, but needed in the lab. If you see something that needs doing (e.g., scheduling a lab cleanup day, or getting gas for the grills before a BBQ), please offer to do it, or just do it (depending on what it is). 
\item Share your skills and knowledge and be open to learning from your peers. 
\item Write your documents in \verb|LaTeX| (or txt or rtf for quicker documents).
\item Organize all code with \verb|git| and use github to share code.
\item Check in with Lizzie before booking your field season travel (and other major travel). 
\item Adjust your schedule to make yourself the most productive (and hopefully) efficient, but plan to be on-campus most work days for at least 6 hours.
\item Check before you schedule holiday time of beyond a few days and be sure to add it to the lab calendar.
\item Share lab resources well. If multiple people need the same resource try to make it work, and if you plan to monopolize a resource for a while, bring it up at scrum first. 
\item Back up your computer! Pester to Lizzie to finalize a lab resource for this.
\item Ask Lizzie about the current computer policy---there is always \$400 CAD available for monitors and peripherals. If you need a laptop we can usuauly find a budget for one, but it does become UBC property when you leave. 
\item Be productive and happy! Look for ways to make the lab a better place for others and suggest ways to make it better for you. 
\end{itemize}

% \newpage
% {\def\section*#1{}
% \bibliography{/Users/Lizzie/Documents/EndnoteRelated/Bibtex/LizzieMainMinimal}
% }

\end{document}



