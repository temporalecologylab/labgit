\documentclass[11pt,letter]{article}
\usepackage[top=1.00in, bottom=1.0in, left=1.1in, right=1.1in]{geometry}
\renewcommand{\baselinestretch}{1.1}
\usepackage{graphicx}
\usepackage{natbib}
\usepackage{amsmath}
\usepackage{hyperref}


\parindent=0pt
\parskip=5pt


\title{Managing and organizing timelines (with Lizzie)}
\author{EM Wolkovich}
\date{\today}

\begin{document}

\maketitle

If you are a trainee in the lab, I may ask you to prepare a timeline. The goal of this is to get you thinking ahead on the tasks and deadlines to finish something on time, and for us both to keep track of your progress. Often that `something' is a PhD and MSc degree (but it could be a paper, grant, executing a short-course etc.). This file outlines some guidelines and important info to remember/know when making a timeline.\\

{\bf Guidelines on how to format your timeline:}
\begin{enumerate}
\item Your timeline can run forward (starting now until some end date) or backward (starting at your end date and running to now), just make sure it covers from now to your end date.  Update it to cover this period as you go. 
\item Your timeline should include the following:
\begin{enumerate}
\item Include a list at the top of what big items you need for your `something' (likely your degree) and when you expect to finish them. This includes things like: proposal approved, comprehensive exam completed, finish first manuscript (give it a very short title), finish second manuscript (see also `A few things to remember' below). 
\item Somewhere below the overview list, include a list for things you need to finish in the next short period (that period is up to you, I recommend picking a period between 1 and 3 months out). 
\item Include a GAANT chart. You can make a simple one with the table command in \LaTeX (example \href{https://www.mjr19.org.uk/IT/gantt_latex.html}{here}) or a way fancier one using the \verb|pgfgantt| package (but only work on a fancy one if it will \emph{really} help you; otherwise, simple is better). 
\end{enumerate}
\item Consider using 2 columns (package \verb|multicol|) for the part of your timeline document that goes through monnth-by-month plans and keep the space tight so it is easier to see the full timeline altogether. 
\item Give each manuscript (chapter in your thesis) a short-hand title you can use to refer to it throughout and avoid referring to `project 1' or `chapter 1' and similar, which is vague (unless you are doing this early and do not know your chapters). 
\item If helpful, include a bulleted or enumerated list of tasks already done somewhere in your timeline file. 
\end{enumerate}


{\bf A few things to remember: }
\begin{enumerate}
\item Your thesis (for MSc and PhD) will be a combination of manuscript chapters. Focus on these are your goal. You can then combine them with a short intro and conclusions to form your full `thesis.' My focus is on your manuscript chapters; I am less concerned about your introduction and conclusion, since few people will ever read those (though some of your committee members may be more focused on these). You also need to pass critical hurdles depending on your degree, such as proposal approved by committee and quals passed. These are major goals to include also. 
\item Remember that I need to review and approve most things you produce and will share with your committee. This includes your proposal, any chapter drafts and the short talk (10-20 minutes) you give at the start of any committee meeting. 
\item Assume a 2-week turnaround time for me to review most written things and that I will review outlines and drafts a minimum of 3-6 times on average before being ready to move on to the next step (e.g., going from outline to writing or going from draft to submission to your committee).
\item Plan to get feedback from the lab, including me, on your committee meeting talk 3-4 weeks before the meeting. 
\item Update your timeline when useful to you or when I ask you to (whichever comes first). 
\item Do not spend so much time on your timeline that you fall behind on tasks on it!
\end{enumerate}


\end{document}
