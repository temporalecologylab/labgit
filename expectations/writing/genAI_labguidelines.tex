\documentclass[11pt]{article}
\usepackage[top=1.00in, bottom=1.0in, left=1in, right=1in]{geometry}
\renewcommand{\baselinestretch}{1.1}
\usepackage{graphicx}
\usepackage{natbib}
\usepackage{amsmath}
\usepackage{parskip}
\usepackage{hyperref}

\def\labelitemi{--}
\parindent=0pt

\begin{document}
\bibliographystyle{/Users/Lizzie/Documents/EndnoteRelated/Bibtex/styles/besjournals}
\renewcommand{\refname}{\CHead{}}

% \setlength{\parindent}{0cm}
% \setlength{\parskip}{5pt}

{\Large My guidelines for use of generative AI with collaborators (especially trainees)}

\emph{5 August 2025}

Generative AI is a new tool that uses large language models, reasoning models and other newer machine learning models to perform a number of tasks, including generating text. While I appreciate the value of this in some regards, I consider writing as a scientist a very important skill. Learning to write well helps you learn to think logically, communicate well and connect with different audiences. I don't believe there is good evidence that using generative AI (e.g., chatGPT, current versions of Grammerly etc.) to write helps you learn to write. Further, I think it makes it harder for me to train you to the best of my ability if you use these products versus share your writing with me. 

I think it is also important to recognize and stay informed of current \href{http://grad.ubc.ca/current-students/student-responsibilities/use-generative-ai}{UBC guidelines} on the use of generativeAI for producing and editing text. In particular, you should understand when, why and how using generative AI to generate text is {\bf plagiarism.}

{\bf My current policy if you are in the lab and/or collaborating with me} is that I ask you to not use generativeAI for any writing you will submit to me.\footnote{These guidelines apply to your writing of text, not your writing of code.}  I am happy to discuss this policy and my reasons for it. It's not a policy that I consider set in stone forever, but I do feel it is the best policy for now for me to be most helpful as someone training others in writing and to meet my own personal standards of writing. 

{\bf If you have asked me to be on your supervisory committee} my rules are:
\begin{enumerate}
\item If you want to use generative AI to help you generate ideas, text or structure for your writing. I am not currently a good choice for your supervisory committee. 
\item I’d prefer you not use generative AI to edit your writing at all. I realize that means I will read text with some grammatical errors and that is okay. 
\item I understand that you may want to use it to edit your grammar (though I do encourage you to find out if there are tools that do not use generative AI that could do this for you). If you want to use generative AI as a tool to help you edit your writing, you must promise me you will not use generative AI beyond this and show me that you know the different levels of using generative AI and how to properly disclose them. Thus, you need to:
\begin{enumerate}
\item Make up a short (say, 1-3 pages) document showing examples of both how you use AI to edit your writing \emph{and} examples where using it without quotation marks would be considered plagiarism. Please also include examples where it is editing your language and not your grammar, and confirm you will not do this in writing you submit to me
\item You also need to find a way to share or record all of the changes made so you can document them.
\item Acknowledge that you know that using generative AI for writing that I am reviewing without disclosing it to me is \emph{academic misconduct} to me. (So please do not do this.)
\end{enumerate}
\end{enumerate}

\newpage
\emph{About coding and other uses of generative AI...} 

This policy applies to writing of text for proposals, manuscripts, grants etc. but does not apply to writing code. I consider writing code different for many reasons, including that I am not asked to read your code line by line (instead I evaluate the output) and that you can (and should) test your code in ways you cannot test your writing. That doesn't mean I endorse using generative AI for coding. In my experience, it is not a great way to learn to code unless you have other structured ways to learn supplement coding help from generative AI. And you should invest more in learning unit testing and other ways to test code, especially if you are using generative AI heavily for coding. 

As in all of graduate school, I encourage you to think about the best ways to learn and structure your learning so that you build skills (not just get the product and/or degree).

% Relatedly, I don't think generative AI is necessarily a great tool to develop your ideas, hypotheses etc. and you should think 

\end{document}

