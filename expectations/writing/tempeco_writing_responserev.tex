\documentclass[11pt,letter]{article}
\usepackage[top=1.00in, bottom=1.0in, left=1.1in, right=1.1in]{geometry}
\renewcommand{\baselinestretch}{1.1}
\usepackage{graphicx}
\usepackage{natbib}
\usepackage{amsmath}
\usepackage{parskip}
\usepackage{hyperref}


\def\labelitemi{--}
\parindent=0pt

\begin{document}
\bibliographystyle{/Users/Lizzie/Documents/EndnoteRelated/Bibtex/styles/besjournals}
\renewcommand{\refname}{\CHead{}}

\title{Writing response to review letters \\ in the Temporal Ecology Lab}
\author{EM Wolkovich}
\date{\today}

\begin{document}

\maketitle

\section{A few quick notes}
There's lots of other advice out there in books (for example, \emph{Write it Up!} though note that I disagree that you cannot set aside harsh reviews for a couple days to give yourself some space, you can! Just don't wait too long; and Stephen Heard's book on writing) and this blog post is quite good: \url{http://ecoevoevoeco.blogspot.com/2015/05/how-to-respond-to-reviewers.html}, which is modified from a talk that Jonathan Davies and Andrew Hendry used to give to grad students in Biology at McGill.

The lab maintains a set of cover letters from lab members, including a whole set of revision letters in \verb|labshadow/resources|.

\section{How to write response to review letters}

I usually write a two-part letter. The first is a one-page overview that often re-uses a good portion of the original cover letter, but the middle is the response to review overview. The second part is a multi-page response to reviewers with the full review written out, and my response to each part clearly delineated. 

Your goal in responding to reviewers is to calmly and rationally address the reviewer comments. While often my initial reaction is annoyance at many comments, I am often impressed by how much reviewers' improve the manuscript. I try to always remember that each reviewer likely represents many possible readers and if I can help that reviewer better understand or appreciate my paper, then I am reaching many more readers who might have similar reactions. 

This does not mean the reviewer is always right, but it does mean that you should go out of your way to address concerns that are not blatantly wrong or otherwise wack-a-doodle. And when they are wrong, I try to double check I did not confuse the reviewer(s) somehow and so maybe there is something upstream of the comment I can adjust or fix. It does generally mean that comments you see across reviewers are problems with your paper that you need to fix (and it's nice we have a peer review process for this).

You also want to make it clear to the editor and reviewers that \emph{you worked to address all concerns.} This means you should not generally focus or lead with places where you did not exactly do what the reviewer asked. Be sure to review first all the changes you made. When I make a lot of changes I often highlight in the overview which parts were changed---abstract, intro, discussion, which figures, title---mention them. Then try to explain the few changes not exactly made. 

This goes for both any overviews and individual responses. While it feels natural to say `we didn't do this, here's why, but we did these smaller changes,' more often it's better to say `we changed this and this; we didn't change this and here's why.' That said, when you're not making a bigger change you may want to start with an overview of your understanding of the concern, then why you did not address it as requested, the clearly lead off a paragraph with what you {\bf did change.} I am to just make all the changes because of my above reasoning and the reality that editors do not want to approve a set of revisions without a good portion of concerns addressed. So, when you do not address a concern, make it the exception and not the rule and make a very good, rational, and calm case for it. 

Some of the earliest advice I got always has stuck with me. This person said, "as an editor looking at a response to review letter, I am hoping to see a lot of responses that say `Agreed.' `Done, here's how ...' or just `Done.' That's a letter you hope to write, and I am happy to read."


\section{Miscellaneous advice}

I usually thank the reviewer for (or otherwise acknowledge) positive comments. 

Be sure to clearly state how you have differentiated your responses from the reviewer comments. 

Make sure your letters are well written with no errors, no run-on sentences or long paragraphs etc..

If there is emphasis in the reviewers' comments add "(reviewer's emphasis)."

You can use \latex to automatically reference your line numbers. Look for more info at \url{https://github.com/temporalecologylab/labgit/wiki/LaTeX}.

\end{document}

