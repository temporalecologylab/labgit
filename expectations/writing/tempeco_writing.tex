\documentclass[11pt,letter]{article}
\usepackage[top=1.00in, bottom=1.0in, left=1.1in, right=1.1in]{geometry}
\renewcommand{\baselinestretch}{1.1}
\usepackage{graphicx}
\usepackage{natbib}
\usepackage{amsmath}

\parindent=0pt
\parskip=5pt


\title{Writing with in the Temporal Ecology Lab (with Lizzie)}
\author{EM Wolkovich}
\date{\today}

\begin{document}

\maketitle

\emph{What's this?} This document is designed to review the method I (Lizzie) like to follow in the lab to write papers collaboratively. It covers the process and then some small and large details that make it easier to write a manuscript with me. It follows closely to how I write papers and has worked consistently for me as a first author and as a senior author. I have found this method works best for me to write with others and so can greatly expedite and improve me helping you. 

I realize we all have different writing styles and processes and it's important that you find one that works for you. Please approach this with an open mind and consider it a chance to try my approach and see what parts work for you (if you really hate it, let's discuss it).

\tableofcontents

\section{Preferred process}
\subsection{Start with the results}
The hardest part of the paper to me is figuring out how to tell the story your results want to tell. Usually, papers begin long before you start a manuscript file as you do analyses, plot results, discuss those results and figures with me and others, and try other approaches etc.. Once you've done this for a while, you can start trying to figure out what your main results \emph{are} and \emph{how to organize them}. 

At this point I suggest you start a document with `main results' and `supplemental results' sections. The main results section should encompass a few key points that fit together, be explained with short sentences (or parts of sentences is fine!) and all related figures and statistics. Additional ones you have done but you don't think belong in the main text can go in  `supplemental results.' {\bf Show me these results} when you feel they are in good shape and be prepared {\bf to iterate with me on them.}
\subsection{Outline, outline, outline}
I really like to outline papers. I find it the fastest and best way to work out the structure of a paper, and---especially for me as not the first author---to quickly review and edit your proposed structure. So, please outline your papers! You can and should start this as you work on the results. Here's an example of what the first bit of a results outline could look like:

\begin{enumerate}
\item Forcing, photoperiod and chilling advance budburst and leafout
\begin{enumerate}
\item Chilling was the largest effect (see Fig 1) among cues, but did not vary between treatments of 1.5 and 4 C
\item Forcing was a close second (Fig 2)
\item Photoperiod had a small effect, but it was present across almost all species
\end{enumerate}
\item Species varied in their response to cues ... 
\end{enumerate}

I like to outline so the top part (1 and 2 just above) are basically the topic of a section (or paragraph) and the points I expect to make go in order below. This makes it easy to skim through and see how the outline flows. A good outline eventually has each paragraph and the gist of the sentences within each paragraph (in the above example, `Forcing, photoperiod and chilling advance budburst and leafout' could be a topic sentence to a paragraph with the following three points (1a-1c) in it). 

I know some people hate outlines so I have repeatedly tried to write and collaborate without them, but it has almost never worked. So, please work with me and outline. 

\subsection{Ordering outlines \& getting started}
I prefer generally to outline in this order: results, methods, introduction, discussion. You can often write methods whenever though (ideally this is the easiest section, but with complex models it is not always so easy at all) and it is {\bf especially good to jot down methods while in the field/lab/in silica while you remember them!} so if you want to get started writing earlier---for example, as you're still analyzing your data---then work on your methods! 

Generally expect to {\bf iterate with me on each part of this} (though it can vary, especially when you're working on your $n$th manuscript, but check with me first). This means iterating on your results a fair bit, then methods (usually the fastest), then introduction and then discussion. 

I personally get started on writing outlines by jotting down everything that I may want to say in a paper or section, then grouping what groups together (usually with a group of `doesn't fit ... yet'), organizing within groups, and iterating until I am happy enough---knowing I will {\bf iterate} on the outline itself once semi-fleshed out, and then again. 

I have a colleague who outlines his papers by writing a really good (aka well-organized) talk first.

\subsection{Please try topic sentences}

Some people hate topic sentences. They had them hammered in too much in writing classes or find they constrict their elegant and expressive writing. I can get that. But good scientific writing gets a ton of info into a very small space, with a story that is easy to follow and compelling and it is damn easier to do that with topic sentences. Also, lots of people skim papers so topic sentences allows them to get the general message of your paper even faster. 

The take home here is to please use topic sentences; if you want to break free from them do it very sparingly. Make your outline have the start to topic sentences at the spot that overviews each paragraph, then write out topic sentences for each paragraph as you write the paper. 

How to tell how you're doing? Read your manuscript by just reading each first sentence of each paragraph, does it flow and make sense? If not, revise and improve. Then read your manuscript and underline every possible topic sentence; if you find you have more than one in a paragraph? Then adjust.

\section{Expectations about time}

Writing papers takes longer than you can likely imagine. I have written dozens of them and am still surprised (and disappointed) by this reality. So think of the longest you have ever spent on writing a paper and multiply 10 for your expectations, then know that it could take way longer. Expect many iterations and drafts and try to make peace wit this. The papers that are most cited are generally those with interesting findings \emph{that are well written} so it's worth it to be patient. 

The rate-limiting step to writing papers is likely me, and I recommend you keep it that way. I dream to turn things around in a week (and I used to long ago when the lab started) but my goal is now two weeks and that deadline gets longer if I am traveling, slammed with other manuscripts or other tasks etc.. 

This means you need to build in my slowness into your deadline and try to avoid being really slow yourself. If you're not sure how long it should take to get something back to me---ask! And whenever possible try to keep your turnaround time faster than mine once past the results section. If we're iterating on an outline, and you take six weeks to review and deal with notes, I will have forgotten much of what I wrote and your manuscript and will be even slower when I get it back from you (because my head will be out of the game, so to speak).  

To prevent myself from being even slower, I don't edit everything, every time. I might focus on results text and figure design, then come back to captions later. Try to keep an eye on this (I will often make a note) and encourage me to review parts I may have missed (equations, captions etc.). 

Expect many drafts. You probably got a sense of that already, but I just wanted to reiterate that here. Writing is about iteration, so expect to iterate to improve.

\section{Software}

We generally write our manuscripts in \LaTeX and \verb|Sweave|. They integrate with version control much better and just generally make our work more reproducible. In exceptional cases you can write in Word or Google Docs if you are an undergraduate, but I don't recommend it and you need to convince me it's a good idea first.  

\section{A few more things to know}

This document is very much in development so here let me share a few things I want you to know that are not (yet?) organized to fit elsewhere... 

\begin{enumerate}
\item Revision is how writing gets to be good writing. All initial writing is generally pretty poor. I need to fix my writing by re-reading and re-writing many times, and that's also how I fix our writing (which is to say: don't be shocked when I edit or complain about something I actually wrote).
\item My writing is your writing. If I suggest edits and you like the phrasing, take it. We're collaborators, so all writings and ideas for the manuscript are shared. 
\end{enumerate}


\section{How to get better at writing}

The very best way to become a better writer is to {\bf read more} and {\bf write more}. Full stop. So all of the below are basically additional versions of those two things that you need to do get better at writing. 
\begin{enumerate}
\item Read lots of books on your own. Fiction, non-fiction etc. 
\item Read stuff other than books, such as well-written longer-form magazine articles and blogs are good to. Read lots! You can also read the paper and social media lots---this is good too---but remember the format here is not as similar to other writing. 
\item Read papers from our the lab. I try hard to make sure most are well-written so hopefully they give you a decent sense of pretty good scientific writing.
\item Find papers you like and try to figure out what makes them work. 
\item Read books and articles about writing. You can even re-read them (I do). I keep a list on our lab website.
\item Write every week, ideally almost every day. Write something: your methods, a blog, in a diary. Write!
\item Start or join a writing group. 
\end{enumerate}


\section{`Rules' to scientific writing}

I say `rules' because language, and hopefully science, is always evolving, but there are some general things to remember. Here they are in no real order:

\begin{enumerate}
\item Methods and Results are both usually in the past tense
\item Give units! Give dates of when you did important things like ran a search for the meta-analysis or started an experiment
\item Avoid being too colloquial. For example, I personally dislike `on the other hand' in scientific writing as it feels too colloquial to me. 
\item Equations! Gelman and Hill discusses how to write equations starting on page 262, with some good alternatives. I like to use words instead of letters for indexing when possible (so `sp' and `study' below) and i is often used for the observation-level in statistics, so should not be used otherwise. 
\end{enumerate}


\end{document}

% Not sure what I mean here ...
I go through words and best phrases
