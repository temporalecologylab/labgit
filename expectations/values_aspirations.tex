\documentclass[11pt,a4paper,oneside]{article}
\renewcommand{\baselinestretch}{1}
% \renewcommand*{\thefootnote}{\fnsymbol{footnote}}
\usepackage{sectsty,setspace} 
\usepackage[top=1.00in, bottom=1.0in, left=1in, right=1in]{geometry} 
\usepackage{graphicx}
\usepackage{epstopdf}
\usepackage{amsmath,latexsym,amssymb,wasysym}
\usepackage{natbib}
\usepackage[obeyspaces]{url}
\usepackage{hyperref}
% \usepackage{lineno}
\usepackage{fancyhdr}
\pagestyle{fancy}
\fancyhead[LO]{Lab Values and Aspirations}
\fancyhead[RO]{Wolkovich Lab}

\newenvironment{smitemize}{
\begin{itemize}
  \setlength{\itemsep}{1pt}
  \setlength{\parskip}{0pt}
  \setlength{\parsep}{0pt}}
{\end{itemize}
}

\def\labelitemi{--}

\begin{document}

\title{Lab Values and Aspirations}
\date{}

\maketitle

% \linenumbers

\section*{Overview}

This documents outlines some of the values we hold as a lab and some of the outcomes we aspire to. It functions as a code of conduct, in the sense that we expect lab members to adhere to any standards that are laid out explicitly. We also hope that even when standards are not explicit, lab members will act in accordance with the spirit of the values laid out here, such as kindness, respect, and openness.\\
\\
\noindent
This is a living document, meaning it will be regularly updated and expanded as we learn more about how to do good science while being good to each other. If you want to see it updated, great! Please let Lizzie know.

\section{The Golden rule: Be kind}

Above all, we aim to be kind to each other. This means treating others as we ourselves would want to be treated: with decency, empathy, and compassion. To do so, we must act inclusively and be considerate of others’ experiences and backgrounds, including but not limited to sexual orientation, gender, disability, origin, race, religion, age, socioeconomic class, family situation, citizenship, and political opinion. \textbf{\underline{We must also actively avoid being unkind}. This means that as a lab we will not tolerate harassment, non-consensual physical contact, inappropriate language, exclusionary jokes, harmful pranks, racism, or sexism.}

\section{Be Safe}

\textbf{Ensure the safety of everyone in our lab}, whether on campus, in the field, or in any other social or professional setting. Therefore lab members are expected to:
\begin{itemize}
  \item Follow UBC rules regarding harassment and bullying while conducting lab or field work
  \item Follow departmental and lab safety protocols 
  \item Ensure equal access to transportation, communication devices, and safety equipment 
  \item Be mindful of people’s need for personal time, space, and privacy (especially during shared living)
  \item Be mindful of how others experience public spaces and be open to discussing and managing situations that might arise
\end{itemize}
  
\noindent
Everyone is responsible for mitigating risks and should educate themselves of differential risks prior to the start of lab or field work. If at any time you feel unsafe in the lab or the field, contact Lizzie to find ways to mitigate the risks. For more reading on staying safe in the field see \href{https://www.nature.com/articles/s41559-020-01328-5}{here}.

\section{Show respect}

\textbf{Maintain a respectful lab environment}. Members should show respect to each other by listening with minimal interruptions and also by not dominating conversations. In other words, members should create space for all to contribute and demonstrate open-mindedness about the ideas and beliefs of others. We should respect reasonable time commitments, work schedules, and need to focus. Where possible, we should also aim to create equal opportunities for collaboration and cooperation in lab projects and activities.\\
\\
\noindent
\textbf{Decolonize research}. Our Faculty, EEB, and much of the world in which we live is the product of a colonial history, and we encourage everyone to explore ways to decolonize their research. This can include simple steps, such as including land acknowledgments before talks or presentations as a way of bringing attention to the colonial history of where we work. For more information on the land acknowledgment for UBC see \href{https://vpfo.ubc.ca/2021/02/what-is-a-land-acknowledgement}{here}. We also aim to build meaningful relationships with collaborators through humility, adaptability, and an openness to Indigenous knowledge and values. We recognize that these relationships, however, take time to develop and require reciprocity, in addition to respect. Anyone who is interested in learning more about decolonizing their work is encouraged to do further readings on the topic and participate in additional learning and training on Indigenous history. We recommend reading works such as \href{https://www.facetsjournal.com/doi/10.1139/facets-2020-0005}{Toward reconciliation: 10 Calls to Action to natural scientists working in Canada} and reviewing the \href{http://trc.ca/assets/pdf/Calls_to_Action_English2.pdf}{96 calls to action} offered by the Truth and Reconciliation Commission.

\section{Communicate effectively and fairly}

\textbf{Listen actively} and find other ways to improve the overall experience of communication and understanding for all parties involved. This could mean asking clarifying questions to improve your understanding, summarizing what has been said, or otherwise finding ways to both to improve your understanding and to communicate to others that you value what they’re saying.

\section{Share resources}

\textbf{Share knowledge} openly within the lab and, eventually, with the world. Also, be appreciative of the knowledge that you receive. See the \href{https://github.com/temporalecologylab/labgit/blob/master/datacodemgmt/tempeco_DMP.pdf}{Wolkovich Data Management Policy} document for further details on sharing data with the lab.\\
\\
\noindent
\textbf{Take care of shared spaces/equipment}. We should work to keep shared spaces clean and shared equipment in good working order. We must also share spaces and equipment fairly and equitably. We also aim to share responsibilities for maintaining clean living conditions when sharing accommodations by balancing the allocation of lab and field maintenance tasks.

\section{Engage with your imperfections}

We are all imperfect. Accepting this reality makes it more possible for you to be a better person (e.g., see talk by \href{https://www.youtube.com/watch?v=MbdxeFcQtaU}{Jay Smooth}). Engaging with your imperfections does not mean you should be comfortable or complacent in working towards our lab aspirations here. Instead it means you accept that we can all do better, and we need to engage with and accept that to improve. \\
\\
\noindent
This same point goes for systems (including this lab) and this document. UBC, EEB and academia have a lot of problems and we need to engage with these realities to improve our systems. It also unfortunately means that when things go wrong, we may not yet have all the resources to make them fully better. Likewise, parts of this document may someday seem short-sighted or outdated, and require your help to improve.

\section*{How to report problems}

To change, update, report problems or serious violations of this code of conduct, please contact lab leader Lizzie Wolkovich. She aims to treat most concerns/reports/requests  confidentially, though be aware that she is obligated to report certain forms of misconduct to others, so feel free to ask about confidentiality first. \\
\\
\noindent
If you are not comfortable contacting Lizzie, another place to look for help is UBC’s \href{https://equity.ubc.ca/how-we-can-help/}{Equity and Inclusion Office}. Other possible points of contact include the \href{https://ombudsoffice.ubc.ca/}{Office of the Ombudsperson}.\\
\\
\noindent
If you are concerned about harassment or bullying, see \href{https://bullyingandharassment.ubc.ca/resources/}{UBC’s Harassment and Bullying Policies}.\\
\\
\\
Last Updated: \today

\end{document}



